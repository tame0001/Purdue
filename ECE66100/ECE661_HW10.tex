\documentclass[11pt]{article}

% ------
% LAYOUT
% ------
\textwidth 165mm %
\textheight 230mm %
\oddsidemargin 0mm %
\evensidemargin 0mm %
\topmargin -15mm %
\parindent= 10mm

\usepackage[dvips]{graphicx}
\usepackage{multirow,multicol}
\usepackage[table]{xcolor}

\usepackage{amssymb}
\usepackage{amsfonts}
\usepackage{amsthm}

\usepackage{caption}
\usepackage{subcaption}

\graphicspath{{./ece661_pics/hw10_image/}} % put all your figures here.
\usepackage[shortlabels]{enumitem}
\usepackage{amsmath}
\usepackage{listings}
\usepackage{float}
\usepackage{indentfirst}


\begin{document}
\begin{center}
\Large{\textbf{ECE 661: Homework 10}}

Thirawat Bureetes

(Fall 2020)
\end{center}
	
 
%-----------------------------------------------------------------------------------

\section*{Task1: Projective Stereo Reconstruction}


\subsection*{Fundamental Matrix}

There are two images capturing the same sence: image A and image B. $\mathbf{x}$ and $\mathbf{x}'$ represent the same point in the image A and image B repectively. These two points from two images can be linked together as following.

\begin{align*}
\mathbf{x'}^T\mathbf{F}\mathbf{x} &= 0
\end{align*}

$\mathbf{F}$ is called fundamental matrix. The fundamental matrix is 3 x 3 matrix, therefore, it has 9 elements. However, the Degree of Freedom of $\mathbf{F}$ is 7. So, with 8 manually picked corecespondences between $\mathbf{x}$ and $\mathbf{x}'$, the matrix A can be obtained by solving $\mathbf{A}\mathbf{F} = 0$. Where 

\begin{align*}
\mathbf{A}&= 
\begin{bmatrix} 
x'_ix_i & x'_iy_i & x'_i & y'_ix_i & y'_iy_i & y'_i & x_i & y_i & 1
\end{bmatrix}
\end{align*}

The $\mathbf{F}$ is the eigenvector that corespondence to the minimum eigenvalue of Singular Value Decomposition (SVD) of matrix $\mathbf{A}$. Since $\mathbf{F}$ has rank equal to 2, $\mathbf{F}$ is further decomposed using SVD. The result will be 3 eigenvalues and 3 eigenvectors. Force the mimumum eigenvalue to 0 and recomposite to form $\mathbf{F}$ with rank equal to 2.
 

%-----------------------------------------------------------------------------------

\subsection*{LM Refinement}

To using LM non-linear to optimize $\mathbf{F}$, first, the cost (objective) function has to be declared. Define image A's epipole as $\mathbf{e}$ and image B's epipole as $\mathbf{e'}$. These epipoles are also left and right null vector of $\mathbf{F}$.  

\begin{align*}
\mathbf{F}\mathbf{e} &= 0 \\
\mathbf{e'}^T\mathbf{F} &= 0 \\
\end{align*}

As  $\mathbf{F}$ has rank equal to 2. Epipoles are the eigenvectors (right and left) that correspondence to the smallest eigenvalue. Next, find camera paramenter  $\mathbf{P}$ and  $\mathbf{P'}$. 

\begin{align*}
\mathbf{P} &= [\mathbf{I}|0]  \\
\mathbf{P'} &= [[\mathbf{e'}]_x\mathbf{F}|\mathbf{e'}] 
\end{align*}

Where 

\begin{align*}
[\mathbf{e'}]_x
\begin{bmatrix} 
0 & -e_z & e_y  \\ 
e_z & 0 & -e_x \\ 
-e_y & e_x & 0
\end{bmatrix}
\end{align*}

With each manually picked correspondence paris, the world coordinate ($\mathbf{X}$) is the eigenvector correspondence to the smallest eigenvalue from SVD of $\mathbf{A}$ defined as following

\begin{align*}
\mathbf{A} = 
\begin{bmatrix} 
x_i \mathbf{p}^T_3 - \mathbf{p}^T_1   \\ 
y_i \mathbf{p}^T_3 - \mathbf{p}^T_2   \\ 
x'_i \mathbf{p'}^T_3 - \mathbf{p'}^T_1   \\ 
y'_i \mathbf{p'}^T_3 - \mathbf{p'}^T_2   \\ 
\end{bmatrix}
\end{align*}

Where 

\begin{align*}
\mathbf{P} &= 
\begin{bmatrix} 
\mathbf{p}^T_1 \\
\mathbf{p}^T_2 \\
\mathbf{p}^T_3
\end{bmatrix} \\
\mathbf{P'} &= 
\begin{bmatrix} 
\mathbf{p'}^T_1 \\
\mathbf{p'}^T_2 \\
\mathbf{p'}^T_3
\end{bmatrix}
\end{align*}

The final cost function of an optimization is geomatrix error(distant) as defined below.

\begin{align*}
d^2_{geo} &= \sum ||(\mathbf{x}_i - \mathbf{X}_i  \mathbf{P}_i ) + (\mathbf{x'}_i - \mathbf{X}_i  \mathbf{P'}_i ) ||^2_2
\end{align*}

%-----------------------------------------------------------------------------------

\subsection*{Image Rectification}

To rectifine image, $\mathbf{H}$ and $\mathbf{H'}$ are used for image A and image B respectively. First compute $\mathbf{H'} = \mathbf{T}_2\mathbf{G}\mathbf{R}\mathbf{T}_1$. $\mathbf{T}_1$ is used to translate image into the origin point. $\mathbf{R}$ is used to rotated epipole $\mathbf{e}$ to $\begin{bmatrix}f & 0 & 1\end{bmatrix}^T$. Then  $\mathbf{G}$ translate epipole further to $\begin{bmatrix}f & 0 & 0\end{bmatrix}^T$. Lastly, $\mathbf{T}_2$ translate image to original center.

For $\mathbf{H}$ , starting with $\mathbf{M} = [\mathbf{e'}]_x\mathbf{F}$. $\mathbf{H}_0 = \mathbf{H'}\mathbf{M}$. Then define $\mathbf{H}_a$ as 

\begin{align*}
\mathbf{H}_a = 
\begin{bmatrix} 
a & b & c  \\ 
0 & 1 & 0 \\ 
0 & 0 & 1
\end{bmatrix}
\end{align*}

Use linear least-square to minimize 

\begin{align*}
\sum a \hat{x}_i + b \hat{y}_i + c -  \hat{x'}_i
\end{align*}

Where 

\begin{align*}
\mathbf{\hat{x}}_i &= \mathbf{H}_0 \mathbf{x}_i \\
\mathbf{\hat{x}'}_i &= \mathbf{H'} \mathbf{x'}_i
\end{align*}

Finally, $\mathbf{H} =\mathbf{H}_a \mathbf{H}_0 $


%-----------------------------------------------------------------------------------

\section*{Results}

\begin{figure}[H]
\centering
\includegraphics[width=1\linewidth]{DSC_1603}
\caption{Picture A}
\label{}
\end{figure}

\begin{figure}[H]
\centering
\includegraphics[width=1\linewidth]{DSC_1604}
\caption{Picture B}
\label{}
\end{figure}

\begin{figure}[H]
\centering
\includegraphics[width=1\linewidth]{pic3}
\caption{Picture A with interesting points}
\label{}
\end{figure}

\begin{figure}[H]
\centering
\includegraphics[width=1\linewidth]{pic4}
\caption{Picture B with interesting points}
\label{}
\end{figure}

\begin{figure}[H]
\centering
\includegraphics[height=22cm,keepaspectratio]{pic1}
\caption{Original pictures with corespondance pixels}
\label{}
\end{figure}

\begin{figure}[H]
\centering
\includegraphics[width=1\linewidth]{pic5}
\caption{Rectified picture A with interesting points}
\label{}
\end{figure}

\begin{figure}[H]
\centering
\includegraphics[width=1\linewidth]{pic6}
\caption{Rectified picture B with interesting points}
\label{}
\end{figure}

\begin{figure}[H]
\centering
\includegraphics[height=22cm,keepaspectratio]{pic2}
\caption{Rectified pictures with corespondance pixels}
\label{}
\end{figure}

\begin{figure}[H]
\centering
\includegraphics[width=1\linewidth]{pic7}
\caption{Result of Canny edge detection of image A}
\label{}
\end{figure}

\begin{figure}[H]
\centering
\includegraphics[width=1\linewidth]{pic8}
\caption{Random select 40 points from Canny edge as interesting points}
\label{}
\end{figure}

\begin{figure}[H]
\centering
\includegraphics[width=1\linewidth]{pic9}
\caption{Random select 40 points from Canny edge as interesting points on rectified image}
\label{}
\end{figure}

\begin{figure}[H]
\centering
\includegraphics[height=22cm,keepaspectratio]{pic10}
\caption{Random select 40 points from Canny edge corespondance pixels}
\label{}
\end{figure}


%-----------------------------------------------------------------------------------
\newpage
\section*{Source Code}

\begin{lstlisting}
# Import necessary libraries
import numpy as np
import cv2 as cv
import matplotlib.pyplot as plt
from pathlib import Path
from scipy.optimize import least_squares

path = Path.cwd() / 'ece661_pics' / 'hw10_image'
np.set_printoptions(precision=5)

class Point():
    '''
        Class for storing point or pixel.
    '''
    def __init__(self, x, y):
        self.x = x  
        self.y = y
        self.point = (int(self.x), int(self.y))

    def adjust_width(self, width):
        x = self.x + width
        return (x, self.y)
    
    def adjust_hieght(self, hieght):
        y = self.y + hieght
        return (self.x, y)

    def multiply_with(self, H):
        x = np.array([self.x, self.y, 1])
        new_x = np.dot(H, x)
        x = new_x[0] / new_x[2]
        y = new_x[1] / new_x[2]
        return (Point(x, y))
    
    def vector(self):
        return np.array([self.x, self.y, 1])

    def __repr__(self):
        return f'({self.x:6.1f},{self.y:6.1f})'

class Pair():
    '''
        Class for stroing corespondance pair.
    '''
    def __init__(self, a, b):
        self.a = a
        self.b = b
    
    def __repr__(self):
        return f'{self.a}{self.b}'

class Image():
    ''' 
        Class for storing images.
    '''
    
    def __init__(self, path):
        self.path = path   
        self.load() 

    def load(self):
        filename = f'{self.path.parent}\\{self.path.name}'
        self.image = cv.imread(filename)
        self.image_gray = cv.cvtColor(self.image, cv.COLOR_BGR2GRAY)

    def show(self):
        plt.imshow(cv.cvtColor(self.image, cv.COLOR_BGR2RGB))
   
img_a = Image(path / 'DSC_1603.jpg')
img_b = Image(path / 'DSC_1604.jpg')

def normalize(points):
    '''
        Normalize points. 
    '''
    xs = []
    ys = []
    for point in points:
        xs.append(point.x)
        ys.append(point.y)
    
    xs = np.array(xs)
    ys = np.array(ys)
    mean_x = xs.mean()
    mean_y = ys.mean()
    xs = (xs - mean_x)**2
    ys = (ys - mean_y)**2
    scale = np.sqrt(2) / (np.sqrt((xs + ys)).mean())
    tx = -scale * mean_x
    ty = -scale * mean_y
    T = np.array([
        [scale, 0, tx],
        [0, scale, ty],
        [0, 0, 1]
    ])
    new_points = []
    for point in points:
        new_points.append(point.multiply_with(T))

    return T, new_points

# Manually pick correspondence points
points = [
    # x1 y1 x2 y2
    [785, 115, 826, 63],
    [1648, 56, 1804, 14],
    [1621, 780, 1761, 736],
    [840, 718, 886, 718],
    [544, 775, 593, 795],
    [601, 1275, 655, 1351],
    [1234, 812, 1402, 790],
    [2093, 773, 2151, 701],
    [2045, 1310, 2104, 1210],
    [1248, 1376, 1411, 1365],
    [1964, 714, 1984, 656],
    [2106, 52, 2028, 25]
]
points_a = []
points_b = []
pairs = []
for point in points:
    point_a = Point(point[0], point[1])
    points_a.append(point_a)
    point_b = Point(point[2], point[3])
    points_b.append(point_b)
    pairs.append(Pair(point_a,point_b))    
# Normalize points
T1, points_a_norm = normalize(points_a)
T2, points_b_norm = normalize(points_b)
pairs_norm = [] 
for i in range(len(points)):
    pairs_norm.append(Pair(points_a_norm[i], points_b_norm[i]))

# Construct A matrix for solving AF = 0
A = []
for pair in pairs_norm:
    A.append([
        pair.b.x * pair.a.x,
        pair.b.x * pair.a.y,
        pair.b.x,
        pair.b.y * pair.a.x,
        pair.b.y * pair.a.y,
        pair.b.y,
        pair.a.x,
        pair.a.y,
        1
    ])
A = np.array(A, dtype=np.float)
u, s, vt = np.linalg.svd(A)
# Select the last colume
F = vt.T[:, -1]
F = F.reshape((3, 3))
u, s, vt = np.linalg.svd(F)
# Force last eigen value to 0
s[-1] = 0
F = np.dot(u, np.dot(np.diag(s), vt))
# Denormalize
F = np.dot(T2.T, np.dot(F, T1))
F = F / F[-1, -1]

def compute_camara_params(F):
    '''
     Compute epipoles and camera parameters.
    '''
    # Compute epipoles
    u, s, vt = np.linalg.svd(F)
    e1 = vt.T[:, -1]
    e2 = u[:, -1]
    e1 = e1 / e1[2]
    e2 = e2 / e2[2]
    #Compute camara parameters
    P1 = np.eye(3)
    P1 = np.append(P1, np.zeros((3, 1)), axis=1)
    ex = np.array([
        [0, -e2[2], e2[1]],
        [e2[2], 0, -e2[0]],
        [-e2[1], e2[0], 0]
    ])
    P2 = np.append(np.dot(ex, F), e2.reshape((3, 1)), axis=1)
    return e1, e2, P1, P2

def cost_function(F, pairs):
    '''
        Cost function for LM optimization.
    '''
    F = F.reshape((3, 3))
    e1, e2, P1, P2 = compute_camara_params(F)
    cost_array = []
    for pair in pairs:
        A = np.array([
            pair.a.x*P1[2,:] - P1[0,:],
            pair.a.y*P1[2,:] - P1[1,:],
            pair.b.x*P2[2,:] - P2[0,:],
            pair.b.y*P2[2,:] - P2[1,:]
        ])
        u, s, vt = np.linalg.svd(A)
        X_world = vt.T[:, -1]
        X_world = X_world / np.linalg.norm(X_world)
        P1X = np.dot(P1, X_world)
        P1X = P1X / P1X[-1]
        P2X = np.dot(P2, X_world)
        P2X = P2X / P2X[-1]
        cost = np.linalg.norm(pair.a.vector()-P1X) + np.linalg.norm(pair.b.vector()-P2X)
        cost_array.append(cost)

    return np.array(cost_array)

# LM optimization
res = least_squares(cost_function, F.ravel(), method='lm',
                    args=(pairs, ),
                    max_nfev=5000)
F = res.x
F = F / F[-1]
F = F.reshape((3,3))
u, s, vt = np.linalg.svd(F)
# Force last eigen value to 0
s[-1] = 0
F = np.dot(u, np.dot(np.diag(s), vt))

def compute_homography(F, pairs, img1, img2):

    e1, e2, P1, P2 = compute_camara_params(F)
    h, w = img1.image_gray.shape
    # Compute G
    angle = np.arctan((h/2-e2[1]) / (e2[0]-w/2))
    f = np.cos(angle)*(e2[0]-w/2) - np.sin(angle)*(e2[1]-h/2)
    G = np.eye(3)
    G[2, 0] = -1/f
    # Compute R
    R = np.zeros((3, 3))
    R[0, 0] = np.cos(angle)
    R[0, 1] = -np.sin(angle)
    R[1, 0] = np.sin(angle)
    R[1, 1] = np.cos(angle)
    R[2, 2] = 1
    # Compute T
    T = np.eye(3)
    T[0, 2] = -w/2
    T[1, 2] = -h/2
    # Compute H2
    H2 = np.dot(G, np.dot(R, T))
    center = Point(w/2, h/2)
    center = center.multiply_with(H2)
    T2 = np.eye(3)
    T2[0, 2] = w/2 - center.x
    T2[1, 2] = h/2 - center.y
    H2 = np.dot(T2, H2)
    H2 = H2 / H2 [2, 2]
    # Compute H0
    ex = np.array([
        [0, -e2[2], e2[1]],
        [e2[2], 0, -e2[0]],
        [-e2[1], e2[0], 0]
    ])
    E = np.array([e2, e2, e2]).T
    M = np.dot(ex, F) + E
    H0 = np.dot(H2, M)
    # Compute H1
    A = []
    b = []
    for pair in pairs:
        new_a = pair.a.multiply_with(H0)
        A.append(new_a.vector())
        new_b = pair.b.multiply_with(H2)
        b.append(new_b.vector())
    A = np.array(A)
    b = np.array(b)[:, 0]
    x = np.dot(np.linalg.pinv(A), b)
    HA = np.eye(3)
    HA[0, :] = x
    H1 = np.dot(HA, H0)
    center = Point(w/2, h/2)
    center = center.multiply_with(H1)
    T1 = np.eye(3)
    T1[0, 2] = w/2 - center.x
    T1[1, 2] = h/2 - center.y
    H1 = np.dot(T2, H1)
    H1 = H1 / H1 [2, 2]

    F_rec = np.dot(np.linalg.pinv(H2.T), np.dot(F, np.linalg.pinv(H1)))
    return F_rec, H1, H2

F_rec, H1, H2 = compute_homography(F, pairs, img1, img2)

# Draw line for two images
radius = 10
thickness = 10
h, w = img1.image_gray.shape
canvas = cv.hconcat([img_a.image, img_b.image])
for pair in pairs:
    color = (0, 0, 255)
    cv.circle(canvas, pair.a.point, radius, color, thickness)
    cv.circle(canvas, pair.b.adjust_width(w), radius, color, thickness)
    cv.line(canvas, pair.a.point, pair.b.adjust_width(w), color, thickness)
plt.imshow(cv.cvtColor(canvas, cv.COLOR_BGR2RGB))
savename = f"{path / 'pic1.png'}"
cv.imwrite(savename, canvas.astype(np.int))

def rectify_image(img, H, points):
    '''
        Compute retified image and points.
    '''
    h, w = img.image_gray.shape
    top_left = Point(0, 0)
    top_right = Point(0, w)
    bot_left = Point(h, 0)
    bot_right = Point(h, w)

    top_left = top_left.multiply_with(H)
    top_right = top_right.multiply_with(H)
    bot_left = bot_left.multiply_with(H)
    bot_right = bot_right.multiply_with(H)

    min_x = np.min([top_left.x, top_right.x, bot_left.x, bot_right.x])
    max_x = np.max([top_left.x, top_right.x, bot_left.x, bot_right.x])
    min_y = np.min([top_left.y, top_right.y, bot_left.y, bot_right.y])
    max_y = np.max([top_left.y, top_right.y, bot_left.y, bot_right.y])

    len_x = max_x - min_x
    len_y = max_y - min_y
    
    H = np.linalg.inv(H)
    canvas = np.zeros((int(len_x), int(len_y), 3), dtype=np.uint8)

    for i in range(canvas.shape[1]):
        for j in range(canvas.shape[0]):
            pixel = Point(i+min_x, j+min_y)
            pixel = pixel.multiply_with(H)
            if 0 <= pixel.x < w and 0 <= pixel.y < h:
                canvas[j, i, :] = img.image[int(pixel.y), int(pixel.x), :]

    H_inv = np.linalg.inv(H)
    points_rectified = []
    for point in points:
        point_rectified = point.multiply_with(H_inv)
        x = int(point_rectified.x - min_x)
        y = int(point_rectified.y - min_y)
        points_rectified.append(Point(x, y))
    
    return canvas, points_rectified

img_a_rectifed, points_a_rectified =  rectify_image(img_a, H1, points_a)
img_b_rectifed, points_b_rectified =  rectify_image(img_b, H2, points_b)

pairs_rectified = []
for i in range(len(pairs)):
    pairs_rectified.append(Pair(points_a_rectified[i], points_b_rectified[i]))

# Draw line for two images after rectified
radius = 10
thickness = 10
color = (0, 0, 255)
h_a, w_a, _ = img_a_rectifed.shape
h_b, w_b, _ = img_b_rectifed.shape
max_h = max(h_a, h_b)
canvas = np.zeros((max_h, w_a+w_b, 3), dtype=np.uint8)
canvas[:h_a, :w_a, :] = img_a_rectifed
canvas[:h_b, w_a:, :] = img_b_rectifed
for pair in pairs_rectified:
    cv.circle(canvas, pair.a.point, radius, color, thickness)
    cv.circle(canvas, pair.b.adjust_width(w), radius, color, thickness)
    cv.line(canvas, pair.a.point, pair.b.adjust_width(w), color, thickness)
plt.imshow(cv.cvtColor(canvas, cv.COLOR_BGR2RGB))
savename = f"{path / 'pic2.png'}"
cv.imwrite(savename, canvas.astype(np.int))

def mark_points(img, points, savename):
    radius = 10
    thickness = 10
    color = (0, 0, 255)
    canvas = img.copy()
    for point in points:
        cv.circle(canvas, point.point, radius, color, thickness)
    filename = f'{path / savename}'
    cv.imwrite(filename, canvas.astype(np.int))
mark_points(img_a.image, points_a, 'pic3.png')
mark_points(img_b.image, points_b, 'pic4.png')
mark_points(img_a_rectifed, points_a_rectified, 'pic5.png')
mark_points(img_b_rectifed, points_b_rectified, 'pic6.png')

edges_a = cv.Canny(img_a.image, 100, 200)
plt.imshow(edges_a, cmap = 'gray')
filename = f'{path / "pic7.png"}'
cv.imwrite(filename, edges_a.astype(np.uint8))
h, w = edges_a.shape
edge_points_a = []
for i in range(w):
    for j in range(h):
        if edges_a[j, i] == 255:
            edge_points_a.append(Point(i, j))

sample_edges = random.sample(edge_points_a, 40)
mark_points(img_a.image, sample_edges, 'pic8.png')
img_a_rectifed, sample_edges_rectified =  rectify_image(img_a, H1, sample_edges)
mark_points(img_a_rectifed, sample_edges_rectified, 'pic9.png')

def extract_window(image, point, width):
    return image[point.y - width : point.y + width+1, point.x - width : point.x + width+1]

def compute_ncc(img_a, img_b, points_a):
    image_a_gray = cv.cvtColor(img_a, cv.COLOR_BGR2GRAY)
    image_b_gray = cv.cvtColor(img_b, cv.COLOR_BGR2GRAY)
    pairs = []

    # 21 x 21 pixel window. 
    width = 10 # 21 / 2
    for point in points_a:
        ncc_max = 0
        best_candidate = None
        window_a = extract_window(image_a_gray, point, width)
        candidates = []
        h, w = image_b_gray.shape

        for i in range(width, w-width):
            for j in range(point.y-20, point.y+30):
                candidates.append(Point(i, j))

        for candidate in candidates:
            window_b = extract_window(image_b_gray, candidate, width)
            mean_a = np.mean(window_a)
            mean_b = np.mean(window_b)
            window_a_new = window_a - mean_a
            window_b_new = window_b - mean_b
            num = np.sum(window_a_new * window_b_new)
            den = np.sqrt(np.sum(window_a_new**2) * np.sum(window_b_new**2))
            ncc = num / den
            if ncc > ncc_max:
                ncc_max = ncc
                best_candidate = candidate
        if best_candidate is not None:
            pairs.append(Pair(point, best_candidate))

    return pairs

canny_pairs_rectified = compute_ncc(img_a_rectifed, img_b_rectifed, sample_edges_rectified)

# Draw line for two images after rectified
radius = 10
thickness = 10
color = (0, 0, 255)
h_a, w_a, _ = img_a_rectifed.shape
h_b, w_b, _ = img_b_rectifed.shape
max_h = max(h_a, h_b)
canvas = np.zeros((max_h, w_a+w_b, 3), dtype=np.uint8)
canvas[:h_a, :w_a, :] = img_a_rectifed
canvas[:h_b, w_a:, :] = img_b_rectifed
for pair in canny_pairs_rectified:
    cv.circle(canvas, pair.a.point, radius, color, thickness)
    cv.circle(canvas, pair.b.adjust_width(w_a), radius, color, thickness)
    cv.line(canvas, pair.a.point, pair.b.adjust_width(w_a), color, thickness)
plt.imshow(cv.cvtColor(canvas, cv.COLOR_BGR2RGB))
savename = f"{path / 'pic10.png'}"
cv.imwrite(savename, canvas.astype(np.int))

\end{lstlisting}

%-----------------------------------------------------------------------------------

\end{document}

