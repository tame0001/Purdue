\documentclass[11pt]{article}

% ------
% LAYOUT
% ------
\textwidth 165mm %
\textheight 230mm %
\oddsidemargin 0mm %
\evensidemargin 0mm %
\topmargin -15mm %
\parindent= 10mm

\usepackage[dvips]{graphicx}
\usepackage{multirow,multicol}
\usepackage[table]{xcolor}

\usepackage{amssymb}
\usepackage{amsfonts}
\usepackage{amsthm}

\usepackage{caption}
\usepackage{subcaption}

\graphicspath{{./ece661_pics/}} % put all your figures here.
\usepackage[shortlabels]{enumitem}
\usepackage{amsmath}
\usepackage{listings}
\usepackage{float}


\begin{document}
\begin{center}
\Large{\textbf{ECE 661: Homework 2}}

Thirawat Bureetes

(Fall 2020)
\end{center}
	
 
%-----------------------------------------------------------------------------------

\subsection*{Method to find homography}

To project one image into another image, the homography coordinate can be used to find the corresponding pixels between the pair of images. Let's defind $p = (x, y)$ as a coordinate pixel of the destination image. Its homogeneous coordinates can be defined as $p = \begin{pmatrix} x\\ y\\ 1\end{pmatrix}$. In the similiar way, the source image can be defined as $p' = (x', y')$ or in homogeneous coordinate as $p' = \begin{pmatrix} x'\\ y'\\ 1\end{pmatrix}$. The relationship between these two points are:

\begin{align*}
p' &= \mathbf{H} p \\
\begin{pmatrix} x'\\ y'\\ 1\end{pmatrix} &= 
\mathbf{H} 
\begin{pmatrix} x\\ y\\ 1\end{pmatrix}
\end{align*}

Let's define homography matrix $\mathbf{H}$ as

\begin{align*}
\mathbf{H} &= 
\begin{bmatrix}
h_{11} & h_{12} & h_{13}\\
h_{21} & h_{22} & h_{23} \\
h_{31} & h_{32} & 1
\end{bmatrix} \\
\begin{pmatrix} x'\\ y'\\ 1\end{pmatrix} &= 
\mathbf{H} 
\begin{pmatrix} x\\ y\\ 1\end{pmatrix} \\
\begin{pmatrix} x'\\ y'\\ 1\end{pmatrix} &= 
\begin{bmatrix}
h_{11} & h_{12} & h_{13}\\
h_{21} & h_{22} & h_{23} \\
h_{31} & h_{32} & 1
\end{bmatrix}
\begin{pmatrix} x\\ y\\ 1\end{pmatrix} \\
x' &= h_{11}x + h_{12}y + h_{13}\\
y' &= h_{21}x + h_{22}y + h_{23}\\
1 &= h_{31}x + h_{32}y + 1\\
\end{align*}

Any homogeneous coordinates $\begin{pmatrix} x_h\\ y_h\\ z_h\end{pmatrix}$ is coresponding to physical coordinate $(x_p,y_p)$ following these relationships:
\begin{align*}
x_p &= \frac{x_h}{z_h}\\
y_p &= \frac{y_h}{z_h}\\
\end{align*}

From this reletionship, we get:
\begin{align*}
\frac{x'}{1} &= \frac{ h_{11}x + h_{12}y + h_{13}}{h_{31}x + h_{32}y + 1}\\
x'(h_{31}x + h_{32}y + 1) &= h_{11}x + h_{12}y + h_{13} \\
x' &= h_{11}x + h_{12}y + h_{13} - h_{31}xx' -  h_{32}yx' \\
\frac{y'}{1} &= \frac{ h_{21}x + h_{22}y + h_{23}}{h_{31}x + h_{32}y + 1}\\
y'(h_{31}x + h_{32}y + 1) &= h_{21}x + h_{22}y + h_{23} \\
y' &= h_{21}x + h_{22}y + h_{23} - h_{31}xy' -  h_{32}yy' \\
\end{align*}

As a result, there are 8 unknown variables. To find the values of these unknown, at least 8 equations are needed. As each pixel has both $x$ and $y$, 4 pairs of corresponding are needed. The final system of linear equation is as below.
\begin{align*}
\begin{bmatrix}
x_1^{'}\\ x_2^{'}\\ x_3^{'} \\ x_4^{'} \\ y_1^{'} \\ y_2^{'} \\ y_3^{'} \\ y_4^{'}
\end{bmatrix} &=
\begin{bmatrix}
x_{1} & y_{1} & 1  & 0 & 0 & 0 & -x_{1}x_1^{'} & -y_{1}x_1^{'}\\ 
x_{2} & y_{2} & 1  & 0 & 0 & 0 & -x_{2}x_2^{'} & -y_{2}x_2^{'}\\ 
x_{3} & y_{3} & 1  & 0 & 0 & 0 & -x_{3}x_3^{'} & -y_{3}x_3^{'}\\ 
x_{4} & y_{4} & 1  & 0 & 0 & 0 & -x_{4}x_4^{'} & -y_{4}x_4^{'}\\ 
0 & 0 & 0 & x_{1} & y_{1} & 1 & -x_{1}y_1^{'} & -y_{1}y_1^{'}\\ 
0 & 0 & 0 & x_{2} & y_{2} & 1 & -x_{2}y_2^{'} & -y_{2}y_2^{'}\\ 
0 & 0 & 0 & x_{3} & y_{3} & 1 & -x_{3}y_3^{'} & -y_{3}y_3^{'}\\ 
0 & 0 & 0 & x_{4} & y_{4} & 1 & -x_{4}y_4^{'} & -y_{4}y_4^{'}
\end{bmatrix}
\begin{bmatrix}
h_{11} \\ h_{12}\\ h_{13} \\ h_{21} \\ h_{22} \\ h_{23} \\ h_{31} \\ h_{32}
\end{bmatrix} \\
\begin{bmatrix}
h_{11} \\ h_{12}\\ h_{13} \\ h_{21} \\ h_{22} \\ h_{23} \\ h_{31} \\ h_{32}
\end{bmatrix} &=
\begin{bmatrix}
x_{1} & y_{1} & 1  & 0 & 0 & 0 & -x_{1}x_1^{'} & -y_{1}x_1^{'}\\ 
x_{2} & y_{2} & 1  & 0 & 0 & 0 & -x_{2}x_2^{'} & -y_{2}x_2^{'}\\ 
x_{3} & y_{3} & 1  & 0 & 0 & 0 & -x_{3}x_3^{'} & -y_{3}x_3^{'}\\ 
x_{4} & y_{4} & 1  & 0 & 0 & 0 & -x_{4}x_4^{'} & -y_{4}x_4^{'}\\ 
0 & 0 & 0 & x_{1} & y_{1} & 1 & -x_{1}y_1^{'} & -y_{1}y_1^{'}\\ 
0 & 0 & 0 & x_{2} & y_{2} & 1 & -x_{2}y_2^{'} & -y_{2}y_2^{'}\\ 
0 & 0 & 0 & x_{3} & y_{3} & 1 & -x_{3}y_3^{'} & -y_{3}y_3^{'}\\ 
0 & 0 & 0 & x_{4} & y_{4} & 1 & -x_{4}y_4^{'} & -y_{4}y_4^{'}
\end{bmatrix}^{-1}
\begin{bmatrix}
x_1^{'}\\ x_2^{'}\\ x_3^{'} \\ x_4^{'} \\ y_1^{'} \\ y_2^{'} \\ y_3^{'} \\ y_4^{'}
\end{bmatrix}
\end{align*}

%-----------------------------------------------------------------------------------

\subsection*{Task 1}

\begin{figure}[H]
\centering
\includegraphics[width=1\linewidth]{painting1}
\caption{Frame picture A}
\label{}
\end{figure}

\begin{figure}[H]
\centering
\includegraphics[width=1\linewidth]{painting2}
\caption{Frame picture B}
\label{}
\end{figure}

\begin{figure}[H]
\centering
\includegraphics[width=1\linewidth]{painting3}
\caption{Frame picture C}
\label{}
\end{figure}

\begin{figure}[H]
\centering
\includegraphics[width=1\linewidth]{kittens}
\caption{Image D}
\label{}
\end{figure}

\begin{figure}[H]
\begin{subfigure}{.5\textwidth}
  \centering
  \includegraphics[width=1\linewidth]{painting1}
  \caption{Original image}
  \label{}
\end{subfigure}
\begin{subfigure}{.5\textwidth}
  \centering
  \includegraphics[width=1\linewidth]{hw2_task1_1_1}
  \caption{Projected image}
  \label{}
\end{subfigure}
\caption{Projected image D into frame of image A. Coordinates used for homography (298, 511), (1774, 357), (1686, 1826), (238, 1605).}
\end{figure}

\begin{figure}[H]
\begin{subfigure}{.5\textwidth}
  \centering
  \includegraphics[width=1\linewidth]{painting2}
  \caption{Original image}
  \label{}
\end{subfigure}
\begin{subfigure}{.5\textwidth}
  \centering
  \includegraphics[width=1\linewidth]{hw2_task1_1_2}
  \caption{Projected image}
  \label{}
\end{subfigure}
\caption{Projected image D into frame of image B. Coordinates used for homography (342, 690), (1888, 750), (1886, 2002), (334, 2334).}
\end{figure}

\begin{figure}[H]
\begin{subfigure}{.5\textwidth}
  \centering
  \includegraphics[width=1\linewidth]{painting3}
  \caption{Original image}
  \label{}
\end{subfigure}
\begin{subfigure}{.5\textwidth}
  \centering
  \includegraphics[width=1\linewidth]{hw2_task1_1_3}
  \caption{Projected image}
  \label{}
\end{subfigure}
\caption{Projected image D into frame of image C.  Coordinates used for homography (106, 444), (1224, 306), (1098, 1862), (120, 1364).}
\end{figure}

\begin{figure}[H]
\centering
\includegraphics[width=1\linewidth]{hw2_task1_2_1}
\caption{Projected image A into frame of image C}
\label{}
\end{figure}

%-----------------------------------------------------------------------------------

\subsection*{Task 2}

\begin{figure}[H]
\centering
\includegraphics[width=1\linewidth]{DSC_1473}
\caption{Plane image E}
\label{}
\end{figure}

\begin{figure}[H]
\centering
\includegraphics[width=1\linewidth]{DSC_1474}
\caption{Plane image F}
\label{}
\end{figure}

\begin{figure}[H]
\centering
\includegraphics[width=1\linewidth]{DSC_1475}
\caption{Plane image G}
\label{}
\end{figure}

\begin{figure}[H]
\centering
\includegraphics[width=1\linewidth]{manu}
\caption{Image H}
\label{}
\end{figure}

\begin{figure}[H]
\begin{subfigure}{.5\textwidth}
  \centering
  \includegraphics[width=1\linewidth]{DSC_1473}
  \caption{Original image}
  \label{}
\end{subfigure}
\begin{subfigure}{.5\textwidth}
  \centering
  \includegraphics[width=1\linewidth]{hw2_task2_1_1}
  \caption{Projected image}
  \label{}
\end{subfigure}
\caption{Projected image H into frame of image E. Coordinates used for homography (1483, 537), (3999, 457), (4112, 2362), (1260, 2405).}
\end{figure}

\begin{figure}[H]
\begin{subfigure}{.5\textwidth}
  \centering
  \includegraphics[width=1\linewidth]{DSC_1474}
  \caption{Original image}
  \label{}
\end{subfigure}
\begin{subfigure}{.5\textwidth}
  \centering
  \includegraphics[width=1\linewidth]{hw2_task2_1_2}
  \caption{Projected image}
  \label{}
\end{subfigure}
\caption{Projected image H into frame of image F. Coordinates used for homography (1808, 855), (3905, 83), (4106, 2295), (1648, 2408).}
\end{figure}

\begin{figure}[H]
\begin{subfigure}{.5\textwidth}
  \centering
  \includegraphics[width=1\linewidth]{DSC_1475}
  \caption{Original image}
  \label{}
\end{subfigure}
\begin{subfigure}{.5\textwidth}
  \centering
  \includegraphics[width=1\linewidth]{hw2_task2_1_3}
  \caption{Projected image}
  \label{}
\end{subfigure}
\caption{Projected image H into frame of image G. Coordinates used for homography (2360, 194), (4656, 521), (4935, 2295), (2376, 2355).}
\end{figure}

\begin{figure}[H]
\centering
\includegraphics[width=1\linewidth]{hw2_task2_2_1}
\caption{Projected image E into frame of image G}
\label{}
\end{figure}

%-----------------------------------------------------------------------------------

\subsection*{Code for part 1}

\begin{lstlisting}[language=Python, showstringspaces=false]
# Import necessary libraries
import numpy as np
import cv2 as cv
import matplotlib.pyplot as plt  

def compute_h (coordinates):
    ''' This function will compute the homography matrix
        Input: A list of 4 lists following this format [x, y, x', y']
        where x and y are pixel from destination image
        and x' and y' are the pixel from source image
        Output: A homograply 3x3 matrix'''

        # Create Ax=b where x is the solution
    A = np.zeros((8, 8))
    b = np.zeros((8, 1))
    for i in range(len(coordinates)):
        A[i][0] = coordinates[i][0]
        A[i][1] = coordinates[i][1]
        A[i][2] = 1
        A[i][3] = 0
        A[i][4] = 0
        A[i][5] = 0
        A[i][6] = -1*coordinates[i][0]*coordinates[i][2]
        A[i][7] = -1*coordinates[i][1]*coordinates[i][2]

        A[i+4][0] = 0
        A[i+4][1] = 0
        A[i+4][2] = 0
        A[i+4][3] = coordinates[i][0]
        A[i+4][4] = coordinates[i][1]
        A[i+4][5] = 1
        A[i+4][6] = -1*coordinates[i][0]*coordinates[i][3]
        A[i+4][7] = -1*coordinates[i][1]*coordinates[i][3]

        b[i][0] = coordinates[i][2]
        b[i+4][0] = coordinates[i][3]
    
    A_inv = np.linalg.pinv(A)
    H = np.matmul(A_inv, b)
    H = np.append(H, 1)
    H = H.reshape((3, 3))

    return H

def project_image(src_img, dest_img, H):
    ''' This function will project an image to another image base on homography
        Input: 
        src_img = source image
        dest_img = destination image
        H = homography matrix
        Output: 
        projected image

    '''
    result_img = np.zeros(dest_img.shape)
    for x in range(dest_img.shape[0]):
        for y in range(dest_img.shape[1]):
            pixel = np.array([y, x, 1])
            source_pixel = np.matmul(H, pixel)
            proj_y = source_pixel[0] / source_pixel[2]
            proj_x = source_pixel[1] / source_pixel[2]
            if proj_x < 0 or proj_x > src_img.shape[0]-1 or proj_y < 0 or proj_y > src_img.shape[1]-1:
                result_img[x][y] = dest_img[x][y]
            else:
                result_img[x][y] = src_img[int(proj_x)][int(proj_y)]
    result_img = result_img.astype(int)
    return result_img

source_img = cv.imread('kittens.jpeg')
frame_images = {}
frame_image = {
    'filename': 'painting1.jpeg',
    'savename': 'hw2_task1_1_1.png'
}
frame_images['A'] = frame_image
frame_image = {
    'filename': 'painting2.jpeg',
    'savename': 'hw2_task1_1_2.png'
}
frame_images['B'] = frame_image
frame_image = {
    'filename': 'painting3.jpeg',
    'savename': 'hw2_task1_1_3.png'
}
frame_images['C'] = frame_image
frame_images

# For image A
coordinates = [
    [298, 511, 0, 0],
    [1774, 357, 1920, 0],
    [1686, 1826, 1920, 1125],
    [238, 1605, 0, 1125]
]
HDA = compute_h(coordinates)
dest_img = cv.imread(frame_images['A']['filename'])
resultD2A = project_image(source_img, dest_img, HDA)
plt.imshow(resultD2A)
cv.imwrite(frame_images['A']['savename'], resultD2A)

# For image B
coordinates = [
    [342, 690, 0, 0],
    [1888, 750, 1920, 0],
    [1886, 2002, 1920, 1125],
    [334, 2334, 0, 1125]
]
HDB = compute_h(coordinates)
dest_img = cv.imread(frame_images['B']['filename'])
resultD2B = project_image(source_img, dest_img, HDB)
plt.imshow(resultD2B)
cv.imwrite(frame_images['B']['savename'], resultD2B)

# For image C
coordinates = [
    [106, 444, 0, 0],
    [1224, 306, 1920, 0],
    [1098, 1862, 1920, 1125],
    [120, 1364, 0, 1125]
]
HDC = compute_h(coordinates)
dest_img = cv.imread(frame_images['C']['filename'])
resultD2C = project_image(source_img, dest_img, HDC)
plt.imshow(resultD2C)
cv.imwrite(frame_images['C']['savename'], resultD2C)

\end{lstlisting}

%-----------------------------------------------------------------------------------

\subsection*{Code for part 2}

\begin{lstlisting}[language=Python, showstringspaces=false]
# Import necessary libraries
import numpy as np
import cv2 as cv
import matplotlib.pyplot as plt  

def compute_h (coordinates):
    ''' This function will compute the homography matrix
        Input: A list of 4 lists following this format [x, y, x', y']
        where x and y are pixel from destination image
        and x' and y' are the pixel from source image
        Output: A homograply 3x3 matrix'''

        # Create Ax=b where x is the solution
    A = np.zeros((8, 8))
    b = np.zeros((8, 1))
    for i in range(len(coordinates)):
        A[i][0] = coordinates[i][0]
        A[i][1] = coordinates[i][1]
        A[i][2] = 1
        A[i][3] = 0
        A[i][4] = 0
        A[i][5] = 0
        A[i][6] = -1*coordinates[i][0]*coordinates[i][2]
        A[i][7] = -1*coordinates[i][1]*coordinates[i][2]

        A[i+4][0] = 0
        A[i+4][1] = 0
        A[i+4][2] = 0
        A[i+4][3] = coordinates[i][0]
        A[i+4][4] = coordinates[i][1]
        A[i+4][5] = 1
        A[i+4][6] = -1*coordinates[i][0]*coordinates[i][3]
        A[i+4][7] = -1*coordinates[i][1]*coordinates[i][3]

        b[i][0] = coordinates[i][2]
        b[i+4][0] = coordinates[i][3]
    
    A_inv = np.linalg.pinv(A)
    H = np.matmul(A_inv, b)
    H = np.append(H, 1)
    H = H.reshape((3, 3))

    return H

def project_image(src_img, dest_img, H):
    ''' This function will project an image to another image base on homography
        Input: 
        src_img = source image
        dest_img = destination image
        H = homography matrix
        Output: 
        projected image

    '''
    result_img = np.zeros(dest_img.shape)
    for x in range(dest_img.shape[0]):
        for y in range(dest_img.shape[1]):
            pixel = np.array([y, x, 1])
            source_pixel = np.matmul(H, pixel)
            proj_y = source_pixel[0] / source_pixel[2]
            proj_x = source_pixel[1] / source_pixel[2]
            if proj_x < 0 or proj_x > src_img.shape[0]-1 or proj_y < 0 or proj_y > src_img.shape[1]-1:
                result_img[x][y] = [0, 0 ,0]
            else:
                result_img[x][y] = src_img[int(proj_x)][int(proj_y)]
    result_img = result_img.astype(int)
    return result_img

source_img = cv.imread('kittens.jpeg')
frame_images = {}
frame_image = {
    'filename': 'painting1.jpeg',
    'savename': 'hw2_task1_1_1.png'
}
frame_images['A'] = frame_image
frame_image = {
    'filename': 'painting2.jpeg',
    'savename': 'hw2_task1_1_2.png'
}
frame_images['B'] = frame_image
frame_image = {
    'filename': 'painting3.jpeg',
    'savename': 'hw2_task1_1_3.png'
}
frame_images['C'] = frame_image
frame_images

# For image A to B 
coordinates = [
    [342, 690, 298, 511],
    [1888, 750, 1774, 357],
    [1886, 2002, 1686, 1826],
    [334, 2334, 238, 1605]
]
HAB = compute_h(coordinates)

# For image B to C
coordinates = [
    [106, 444, 342, 690],
    [1224, 306, 1888, 750],
    [1098, 1862, 1886, 2002],
    [120, 1364, 334, 2334]
]
HBC = compute_h(coordinates)

HAC = np.matmul(HAB, HBC)
source_img = cv.imread(frame_images['A']['filename'])
dest_img = cv.imread(frame_images['C']['filename'])

resultA2C = project_image(source_img, dest_img, HAC)
plt.imshow(resultA2C)
cv.imwrite('hw2_task1_2_1.png', resultA2C)

\end{lstlisting}

\end{document}

