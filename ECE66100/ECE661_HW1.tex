\documentclass[11pt]{article}

% ------
% LAYOUT
% ------
\textwidth 165mm %
\textheight 230mm %
\oddsidemargin 0mm %
\evensidemargin 0mm %
\topmargin -15mm %
\parindent= 10mm

\usepackage[dvips]{graphicx}
\usepackage{multirow,multicol}
\usepackage[table]{xcolor}

\usepackage{amssymb}
\usepackage{amsfonts}
\usepackage{amsthm}

\usepackage{caption}
\usepackage{subcaption}

\graphicspath{{./ece595_pics/}} % put all your figures here.
\usepackage[shortlabels]{enumitem}
\usepackage{amsmath}
\usepackage{listings}
\usepackage{float}


\begin{document}
\begin{center}
\Large{\textbf{ECE 661: Homework 1}}

Thirawat Bureetes

(Fall 2020)
\end{center}
	
%\subsection*{Question 1}

\begin{enumerate}
 
%-----------------------------------------------------------------------------------

\item 
\noindent\textbf{Answer}

The point $(x, y)^T$ in physical space $\mathbb{R}^2$ and its homogeneous coordinate $(u,v,w)^T$ in representational space $\mathbb{R}^3$ have relationship as below.

\begin{align*}
x &= \frac{u}{w} \\
y &= \frac{v}{w}
\end{align*}

To represent the origin of physical space $\mathbb{R}^2$ $(0,0)^T$ in representational space $\mathbb{R}^3$.

\begin{align*}
x &= \frac{u}{w} = 0\\
y &= \frac{v}{w} = 0
\end{align*}

The equations above are true when $u = 0$, $v = 0$ and $w \neq 0$. So, the origin of physical space $\mathbb{R}^2$ in representational space $\mathbb{R}^3$ is $(0, 0, w)^T$ where $w = \{w|w \in \mathbb{R}, w \neq 0\}$.

%-----------------------------------------------------------------------------------

\item 
\noindent\textbf{Answer}

$[u,v,w]^T$ in representational space $\mathbb{R}^3$ can represent infinity in physical space $\mathbb{R}^2$ when $w \to 0$. Therefore, $(u_1,v_1,0)^T$ and $(u_2,v_2,0)^T$ are both represent the infinity in physical space even $u_1 \neq u_2$ and $v_1 \neq v_2$. So  the infinity in physical space $\mathbb{R}^2$ are not the same.

%-----------------------------------------------------------------------------------

\item 
\noindent\textbf{Answer}

The degenerate conic is defined as $\mathbf{C} = \mathbf{l}\mathbf{m}^T + \mathbf{m}\mathbf{l}^T$. Each term of the RHS is an outer product, therefore the rank of the result is always 1. From rank property $rank(\mathbf{A}+\mathbf{B}) \leq rank(\mathbf{A})+rank(\mathbf{B})$

\begin{align*}
rank(\mathbf{l}\mathbf{m}^T+\mathbf{m}\mathbf{l}^T) &\leq rank(\mathbf{l}\mathbf{m}^T)+rank(\mathbf{m}\mathbf{l}^T) \\
rank(\mathbf{C}) &\leq 1 + 1 \\
&\leq 2
\end{align*}

%-----------------------------------------------------------------------------------
\newpage
\item 
\noindent\textbf{Answer}

\begin{align*}
\mathbf{l}_1 &= 
	\begin{pmatrix}
		0 \\ 0 \\ 1
	\end{pmatrix} 
	\times
	\begin{pmatrix}
		3 \\ 5 \\ 1
	\end{pmatrix}\\
&= 
	\begin{pmatrix}
		-5 \\ 3 \\ 0
	\end{pmatrix}\\
\mathbf{l}_2 &= 
	\begin{pmatrix}
		-3 \\ 4 \\ 1
	\end{pmatrix} 
	\times
	\begin{pmatrix}
		-7 \\ 5 \\ 1
	\end{pmatrix}\\
&= 
	\begin{pmatrix}
		-1 \\ -4 \\ 13
	\end{pmatrix}\\
\mathbf{l}_1 \times  \mathbf{l}_2&= 
	\begin{pmatrix}
		-5 \\ 3 \\ 0
	\end{pmatrix} 
	\times
	\begin{pmatrix}
		-1 \\ -4 \\ 13
	\end{pmatrix}\\
&= 
	\begin{pmatrix}
		39 \\ 65 \\ 23
	\end{pmatrix}\\
\end{align*}

The intersection between $\mathbf{l}_1$ and $\mathbf{l}_2$ is $(\frac{39}{23},\frac{65}{23})^T$ in physical space.

\begin{align*}
\mathbf{l}_{2,new} &= 
	\begin{pmatrix}
		-7 \\ -5 \\ 1
	\end{pmatrix} 
	\times
	\begin{pmatrix}
		7 \\ 5 \\ 1
	\end{pmatrix}\\
&= 
	\begin{pmatrix}
		-10 \\ -14 \\ 0
	\end{pmatrix}\\
\mathbf{l}_1 \times  \mathbf{l}_{2,new}&= 
	\begin{pmatrix}
		-5 \\ 3 \\ 0
	\end{pmatrix} 
	\times
	\begin{pmatrix}
		-10 \\ -14 \\ 0
	\end{pmatrix}\\
&= 
	\begin{pmatrix}
		0 \\ 0 \\ 100
	\end{pmatrix}\\
\end{align*}

The new intersection point is at the origin. Consider that both $\mathbf{l}_1$ and$\mathbf{l}_{2,new}$ pass the origin. So the intersection could be determinded without calculating the result of cross product of lines. There are 2 steps requiered. 1) find $\mathbf{l}_1$. 2) find $\mathbf{l}_{2,new}$. 

%-----------------------------------------------------------------------------------
\newpage
\item 
\noindent\textbf{Answer}

\begin{align*}
\mathbf{l}_1 &= 
	\begin{pmatrix}
		0 \\ 0 \\ 1
	\end{pmatrix} 
	\times
	\begin{pmatrix}
		5 \\ -3 \\ 1
	\end{pmatrix}\\
&= 
	\begin{pmatrix}
		3 \\ 5 \\ 0
	\end{pmatrix}\\
\mathbf{l}_2 &= 
	\begin{pmatrix}
		-5 \\ 0 \\ 1
	\end{pmatrix} 
	\times
	\begin{pmatrix}
		0 \\ -3 \\ 1
	\end{pmatrix}\\
&= 
	\begin{pmatrix}
		3 \\ 5 \\ 15
	\end{pmatrix}\\
\mathbf{l}_1 \times  \mathbf{l}_2&= 
	\begin{pmatrix}
		3 \\ 5 \\ 0
	\end{pmatrix} 
	\times
	\begin{pmatrix}
		3 \\ 5 \\ 15
	\end{pmatrix}\\
&= 
	\begin{pmatrix}
		75 \\ -45 \\ 0
	\end{pmatrix}\\
\end{align*}

The intersection between $\mathbf{l}_1$ and $\mathbf{l}_2$ is at the infinity since $(u,v,0)^T$ represent the infinity. In the otherhand, it means that $\mathbf{l}_1$ and $\mathbf{l}_2$ are parallel lines. Considering the implicit from of line $ax+by+c = 0$, the slope of line is determined by $\frac{-a}{b}$ and the y-intercept is $\frac{-c}{b}$.  $\mathbf{l}_1$ and $\mathbf{l}_2$ has same $a$ and $b$ but different $c$. This means that  $\mathbf{l}_1$ and $\mathbf{l}_2$ have the same slope but different y-interception points, therefore,  $\mathbf{l}_1$ and $\mathbf{l}_2$ are parallel.

%-----------------------------------------------------------------------------------
\item 
\noindent\textbf{Answer}

From the information given, the ellipse's formula can be writen as

\begin{align*}
\frac{(x-x_0)^2}{a^2} +\frac{(y-y_0)^2}{b^2} &= 1\\
\frac{(x-3)^2}{1^2} +\frac{(y-2)^2}{(\frac{1}{2})^2} &= 1\\
\frac{x^2 - 6x + 9}{1} +\frac{y^2 -4y + 4}{\frac{1}{4}} &= 1\\
x^2 - 6x + 9 + 4y^2 - 16y + 16 - 1 &= 0\\
x^2 + 4y^2 - 6x - 16y + 24 &= 0\\
\end{align*}

This is compitible with implicit from of conic $ax^2 + bxy + cy^2 + dx + ey +f = 0$. Rewrite the ellipse into $\mathbf{C}$.

\begin{align*}
\mathbf{C} &= 
\begin{bmatrix}
a & \frac{b}{2} & \frac{d}{2}\\
\frac{b}{2} & c & \frac{e}{2} \\
\frac{d}{2} & \frac{e}{2} & f
\end{bmatrix} \\
&= 
\begin{bmatrix}
1 & 0 & -3\\
0 & 4 & -8 \\
-3 & -8 & 24
\end{bmatrix}
\end{align*}


let $\mathbf{p}$ is an origin point or $(0,0,1)^T$. 

\begin{align*}
\mathbf{l}_{polar} &= \mathbf{C}\mathbf{p} \\
&= 
\begin{bmatrix}
1 & 0 & -3\\
0 & 4 & -8 \\
-3 & -8 & 24
\end{bmatrix}
\begin{pmatrix}
0\\
0\\
1
\end{pmatrix}\\
&= 
\begin{pmatrix}
-3\\
-8\\
24
\end{pmatrix}\\
\end{align*}

The x-axis can be writen in explicit from as $y = 0$ or in implicit from as $(0)x + (1)y + (0) = 0$. So the x-axis can be writen as $\mathbf{l}_x = (0, 1, 0)^T$. For y-axis, it is $x = 0$ or $(1)x + (0)y + (0) = 0$. So $\mathbf{l}_y = (1, 0, 0)^T$.

\begin{align*}
\begin{pmatrix}
-3\\
-8\\
24
\end{pmatrix}
\times
\begin{pmatrix}
0\\
1\\
0
\end{pmatrix}
&= 
\begin{pmatrix}
-24\\
0\\
-3
\end{pmatrix}\\
\begin{pmatrix}
-3\\
-8\\
24
\end{pmatrix}
\times
\begin{pmatrix}
1\\
0\\
0
\end{pmatrix}
&= 
\begin{pmatrix}
0\\
24\\
8
\end{pmatrix}\\
\end{align*}

The intersection between polar line and x-axis is $(8,0)$. The intersection between polar line and y-axis is $(0,3)$.

%-----------------------------------------------------------------------------------
\item 
\noindent\textbf{Answer}

The line $x=\frac{1}{2}$ can be writen as $(1)x + (0)y - \frac{1}{2} = 0$. And the line $y=-\frac{1}{3}$ can be writen as $(0)x + (1)y + \frac{1}{3} = 0$. 

\begin{align*}
\begin{pmatrix}
1\\
0\\
- \frac{1}{2}
\end{pmatrix}
\times
\begin{pmatrix}
0\\
1\\
\frac{1}{3}
\end{pmatrix}
&= 
\begin{pmatrix}
\frac{1}{2}\\
-\frac{1}{3}\\
1
\end{pmatrix}
\end{align*}

The intersection between $x=\frac{1}{2}$ and $y=-\frac{1}{3}$ is $(\frac{1}{2}, -\frac{1}{3})$.

\end{enumerate}


\end{document}

